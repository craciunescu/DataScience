%%%%%%%%%%%%%%%%%%%%%%%%%%%%%%%%%%%%%%%%%%%%%%%%%%%%%%%%%%%%%%%%%%%%%%%%%%%%%%%%
\documentclass[a4paper]{article}

\usepackage[utf8]{inputenc}
\usepackage[english]{babel}
\usepackage{Sweave}

\begin{document}
\title{G4.PL-2}
\author{David Emanuel Craciunescu \and Laura Pérez Medeiro}

\maketitle

%%%%%%%%%%%%%%%%%%%%%%%%%%%%%%%%%%%%%%%%%%%%%%%%%%%%%%%%%%%%%%%%%%%%%%%%%%%%%%%%

\section*{Introduction}

The objective of the second laboratory practice is to teach students how to
perform data association analysis by using R tools and methodologies.

The other objectives this practice has have already been clearly defined in the
informative memory of Practice 1. That kind of introduction will be considered
obvious henceforth.

Just like the previous practice, it will be divided in two main parts:

%%%%%%%%%%%%%%%%%%%%%%%%%%%%%%%%%%%%%%%%%%%%%%%%%%%%%%%%%%%%%%%%%%%%%%%%%%%%%%%%%

\section*{Data association analysis for six shopping baskets}

%%%%%%%%%%%%%%%%%%%%%%%%%%%%%%%%%%%%%%%

\subsection*{Loading packages}
The beginning of the session started with an insight into work-environment
personalization. The first step towards this is searching the path of the
\textit{Rprofile} configuration file is located.

Given that this file can execute arbitrary R code, the professor recommended
that the students load the package \textit{foreign} by default. Even if useful,
it must be noted that this practice is not considered a good practice and it
might be, in some cases, dangerous.

After some exploration and customization the authors have arrived to the
conclusion that a good \textit{Rprofile} customization can be summarized as
follows:

\begin{Schunk}
\begin{Sinput}
# Example Rprofile.

# Things the user might want to change.
options(editor="vim")
options(...)

# Related to interactive prompt.
options(prompt="> ")
options(continue="+ ")

# General options.
options(tab.width = 4)
options(width = 80)
options(graphics.record = TRUE)

# First function (will be executed last)
.first <- function()
{
    library(foreign)
    cat("\nWelcome!")
}

# Last function (will be executed just before closing)
.Last <- function()
{
    cat("\nGoodbye!")
}
\end{Sinput}
\end{Soutput}

%%%%%%%%%%%%%%%%%%%

Once saved, the following command will provide a list of the packages loaded by
default.
\begin{Schunk}
\begin{Sinput}
> getOption("defaultPackages")
\end{Sinput}
\begin{Soutput}
[1] "datasets" "utils" "grDevices" "graphics" "stats" "methods" "foreign"
\end{Soutput}
\end{Schunk}

The following step will make sure the \textit{arules} package is loaded into the
environment.
\begin{Schunk}
\begin{Sinput}
> library()
\end{Sinput}
\end{Schunk}

It cannot be found here, therefore it must be loaded. In order to do this one
can use either of two choices: (1) using global help to search for
\textit{arules} and downloading a \textit{.zip} from the CRAN mirror or
installing it directly using the\textit{install.packages()} command.

The first method would force us to download the compressed package and install
it from that source. This is an effective way of doing so but can be problematic
when versioning is taken into place.

\begin{Schunk}
\begin{Sinput}
> #install.packages('C:\PATH_TO_THE_LIBRARY\arules_1.5-0.zip', repos=NULL)
> #library(arules)
\end{Sinput}
\end{Schunk}

In order to check that the package has been installed correctly, we can use the command:
\begin{Schunk}
\begin{Sinput}
> search()
\end{Sinput}
\begin{Soutput}
[1] ".GlobalEnv"        "package:stats"     "package:graphics" 
[4] "package:grDevices" "package:utils"     "package:datasets" 
[7] "package:methods"   "Autoloads"         "package:base"     
\end{Soutput}
\end{Schunk}

%%%%%%%%%%%%%%%%%%%%%%%%%%%%%%%%%%%%%%%

The other approach is much simpler and the only thing needed is installing it
directly from source.

\begin{Schunk}
\begin{Sinput}
> install.packages("arules")
\end{Sinput}
\begin{Soutput}
package ‘arules’ successfully unpacked and MD5 sums checked

The downloaded binary packages are in
	C:\Users\PATH_TO_THE_LIBRARY
\end{Soutput}
\end{Schunk}

In case one would want to install a version that's currently in development:
\begin{Schunk}
\begin{Sinput}
> #library("devtools")
> #install_github("mhahsler/arules")
\end{Sinput}
\end{Schunk}


%%%%%%%%%%%%%%%%%%%%%%%%%%%%%%%%%%%%%%%%%%%%%%%%%%%%%%%%%%%%%%%%%%%%%%%%%%%%%%%
\subsection*{Loading data}

The data used for the analysis consistes of this matrix, composed by the six
following baskets:
{Bread, water,  milk,   oranges}
{Bread, water,  coffee, milk}
{Bread, water,  milk}
{Bread, coffee, milk}
{Bread, water}
{Milk}


%%%%%%%%%%%%%%%%%%%%%%%%%%%%%%%%%%%%%%%%%%%%%%%%%%%%%%%%%%%%%%%%%%%%%%%%%%%%%%%%%

\section*{Data association analysis for eigth car sales}
This second part will deal with the association analysis of eight car sales.
This section will reprouce the same analysis performed before, applied to
different data and using a support equal or higher than 40\% and a condifence
equal or higher than 90\%.

First, the .txt field must be generate with the data:
\begin{Schunk}
\begin{Sinput}
> data <- paste(
+ 	"X C N B,",
+ 	"X C B S,",
+ 	"X C N,",
+ 	"X N B S,",
+ 	"X C B,",
+ 	"N,",
+ 	"X C B,",
+ 	"S A,",
+ 	sep="\n")
> cat(data)
\end{Sinput}
\begin{Soutput}
X C N B,
X C B S,
X C N,
X N B S,
X C B,
N,
X C B,
S A,
\end{Soutput}
\begin{Sinput}
> write(data,file = "cars.txt")
\end{Sinput}
\end{Schunk}

Then, we read the file in order to create the transactions
\begin{Schunk}
\begin{Sinput}
> transactions= read.transactions(file="cars.txt",rm.duplicates=FALSE, format="basket", sep=",");
